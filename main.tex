\documentclass[acmsmall,nonacm,screen,review]{acmart}
\newif\ifEnableExtend
%\EnableExtendtrue
\EnableExtendfalse

\usepackage[utf8]{inputenc}
\usepackage{url}
\usepackage{color}
\newcommand{\csch}[1]{{\color{red} Christian says: #1}}
\newcommand{\Is}       {:=}
\newcommand{\set}[1]{\left\{ #1\right\}}
\newcommand{\sodass}{\,:\,}
\newcommand{\setGilt}[2]{\left\{ #1\sodass #2\right\}}
\usepackage{amsmath}
\usepackage{amsthm}
\usepackage{xspace}
\usepackage{relsize}

\newtheorem{openproblem}{Open Problem}
\newcommand{\ie}{i.\,e.,\xspace}
\newcommand{\eg}{e.\,g.,\xspace}
\newcommand{\etal}{et~al.\xspace}
\newcommand{\cov}{\term{cov}\xspace}
\newcommand{\term}[1]{\textsl{#1}}
\newcommand{\Comment}[1]{\textsl{#1}}
\DeclareMathOperator{\vol}{vol}
\DeclareMathOperator{\origVol}{origVol}
\DeclareMathOperator{\origDeg}{origDeg}
%%%%%%%%%%%%%%%%%%%%%%%%%%%%%
\setcopyright{none}
\copyrightyear{2021}
\acmYear{2021}
\acmDOI{}
\acmPrice{}
\acmISBN{}

\title{Beginner's Practical: \\ Hypergraph Conductance-based Clustering}
\author{Mariia Meshchaninova}
\email{mariia.meshchaninova@stud.uni-heidelberg.de, Informatik B.Sc. 100\%, 4740180}
\affiliation{%
  \institution{Heidelberg University}
  \streetaddress{Im Neuenheimer Feld 205}
  \city{Heidelberg}
  \state{Baden-Württemberg}
  \country{Germany}
  \postcode{69120}
}


\date{}

\begin{document}

\begin{abstract}
%Insert abstract here.
Hypergraph clustering is a problem of partitioning the nodes of 
a hypergraph into "true clusters" which are densely connected internally 
and sparsely connected externally. Conductance is a metric measuring the 
quality of a cut which is defined as the ratio of the weight of the cut 
to the minimal volume of one side of the cut.
In this project, we implement a hypergraph clustering algorithm that
heuristically tries to minimize the maximum value of the conductance 
over all cuts between clusters to find a high-quality hypergraph 
clustering. The algorithm is based on the existing shared-memory 
parallel algorithm for hypergraph partitioning \texttt{Mt-KaHyPar}.
\end{abstract}
\maketitle

\section{Introduction}
Hypergraph clustering is a problem of partitioning the nodes of a hypergraph
into "true clusters" which are densely connected internally and sparsely
connected externally. It is useful in many applications, such as 
semi-supervised learning \cite{ApplicationLearning}, 
gene expression analysis \cite{ApplicationGeneExpression}
and image segmentation \cite{ApplicationImageSegmentation}. 
Conductance metric, which is defined as the ratio of the weight of a given
cut to the minimal volume of one side of the cut, is widely used in graph 
clustering \cite{GraphConductance2023,GraphConductance2006} but to the best
of our knowledge, it has only been applied as a metric for evaluating the
quality of a hypergraph clustering \cite{HyperSF}. Therefore in this 
project, we implement a heuristic hypergraph clustering algorithm with the
objective of minimizing the maximum value of the conductance over all cuts between
the clusters while partitioning the hypargraph in at most $k$ clusters. 

\noindent Our algorithm is based on the existing shared-memory parallel 
hypergraph $k$-way partitioner \texttt{Mt-KaHyPar} and uses the same 
multilevel approach as the original algorithm. First, the hypergraph is 
coarsened in stages by detecting and merging node commubities, then the 
coarsest hypergraph is partitioned into $k$ clusters, and finally the 
initial partitioning is refined by uncoarsening the hypergraph nodes and 
applying a local search algorithm to improve the given objective. We 
implement two objective functions for the local search algorithm - \textit{conductance\_local} and \textit{conductance\_global} - 
ans compare their performance on a benchmark set \texttt{ibm01...18} 
against a state of the art hypergraph partitioner \texttt{HyperModularity}.

\noindent In following, we define the used concepts in Section~\ref{sec:preliminaries}, 
then we describe related work on hypergraph partitioning and provide an overview
of the existing conductance-based clustering algorithms in Section~\ref{sec:related_work}. In 
Section~\ref{sec:implementation}, we describe the implementation of the new 
objectives for optimizing the conductance. Afterwards, we present the 
experimental results of our implementation in Section~\ref{sec:experiments}. 
Finally, we conclude the paper and discuss future work in Section~\ref{sec:conclusion}.


%\noindent In this project, we implement a heuristic algorithm for hypergraph 
%lustering with the aim of partitioning the nodes of a hypergraph into 
%at most $k$ clusters, while minimising the maximum value of the conductance 
%over all the cuts between the clusters. The algorithm is based on the 
%existing shared-memory parallel hypergraph partitioner \texttt{Mt-KaHyPar}
%and uses the same multilevel approach as the original algorithm. First,
%the hypergraph is coarsened in stages by detecting and merging node
%commubities, then the coarsest hypergraph is partitioned into $k$ clusters, 
%and finally the initial partitioning is refined by uncoarsening the 
%hypergraph nodes and applying a local search algorithm to improve the given 
%objective. We implement two objective functions for the local search 
%algorithm - \textit{conductance\_local} and \textit{conductance\_global} - 
%ans compare their performance on a benchmark set of instances against



%performance on large benchmark sets of instances against other state-of-the-art 
%hypergraph partitioners \cite{MtKaHyPar2020}. The algorithm follows a multilevel
%approach by in stages coarsening a given hypergraph via detecting and merging 
%communities of the nodes, then partitioning the coarsest hypergraph into $k$ 
%clusters, and then iteratively refining the initial partitioning by uncoarsening
%the hypergraph nodes and applying a local search algorithm to improve the given 
%objective. The original algorithms supports several objectives, such as minimizing 
%the weight of the cut (\textit{cut} metric) or the sum of the external degrees of 
%the clusters (\textit{soed} metric). In this project, we implement two additional 
%objective functions for the \texttt{Mt-KaHyPar} algorithm to minimize the overall 
%maxmal conductance of a cut inbetween the clusters: \textit{conductance\_local} 
%and \textit{conductance\_global}.


\section{Preliminaries}
\label{sec:preliminaries}

In this section, we give an overview of the used concepts and notations and 
define conductance of a hypergraph.

\textbf{Hypergraph} is a generalization of a graph, where a hyperedge connects 
one, two or more nodes. Formally, a hypergraph $H = (V, E, c, \omega)$ consists 
of a non-empty set of nodes $V$, a set of hyperedges (also called \textit{nets}
and simply \textit{edges}) $E \subseteq 2^V$ and weight functions for hypernodes 
and hyperedges $c: V \to \mathbb{N}$, $\omega: E \to \mathbb{N}$. If case of 
unweighted nodes or hyperedges, we set $c(v) = 1$ for all $v \in V$ or 
$\omega(e) = 1$ for all $e \in E$, respectively. 

\textbf{$k$-way clustering} of a hypergraph $H = (V, E, c, \omega)$ is a
partitioning of the set of nodes $V$ into $k$ disjoint subsets 
$V_1, \dots, V_k$ called \textit{clusters}. Empty clusters are allowed, so
$k$ is only an upper bound on the number of clusters. The algorithm is allowed 
to eventially move all the nodes of a cluster to other clusters if it would be
beneficial for the objective function. 

\textbf{Cut} $\partial S$ of a hypergraph $H = (V, E, c, \omega)$ is partitioning
of the set of nodes $V$ into two disjoint non-empty subsets $S$ and 
$V \backslash S$. A hyperedge $e \in E$ is called a \textbf{cutting edge} of 
the cut $\partial S$ if it has at least one pin in $S$ and at least one pin in
$V \backslash S$. The weight of the cut is defined as the sum of the weights of
the cutting edges of the cut $\partial S$: 
\[\omega(\partial S) = \sum_{e \in \partial S} \omega(e)\]
Therefore In case of unweighted hyperedges, the weight of the cut is equal to 
the number of cutting edges of the cut.

\textbf{Pin} of a hyperedge $e \in E$ is a node $v \in V$ such that $v \in e$.
A hyperedge with only one pin is called a \textbf{sinle-pin net}.
In the implementation, we use number of pins $PinNum_i(e)$ of a hyperedge 
$e \in E$ in a cluster $V_i$ to decide whether to move a node $v$ from one
cluster $V_i$ to another cluster $V_j$.

\textbf{Weighted degree} of a node $v \in V$ is defined the same way in 
hypergraphs as in graphs. It is the sum of the weights of all hyperedges 
$e \in E$ that contain the node $v$: 
\[\deg_\omega(v) = \sum_{e\in E: v \in E} \omega(e)\]
As during the coarsening phase of the algorithm we merge nodes, we also use
\textbf{original weighted degree} of a node $v \in V$  $\origDeg_\omega(v)$ 
which is defined as the sum of the weighted degrees of all nodes $v' \in V$ 
that were merged into $v$ during the coarsening phase of the algorithm.

\textbf{Volume} of a hypergraph $H = (V, E, c, \omega)$ is defined as the 
sum of the weighted degrees of all nodes $v \in V$. Analogously, we define the \textbf{volume of a cluster} $V_i$ as the sum 
of the weighted degrees of all nodes $v \in V_i$:
\[\vol(H) = \sum_{v \in V} \deg_\omega(v) \ \ \ \ \ \ \
\vol(V_i) = \sum_{v \in V_i} \deg_\omega(v)\]
And for the same reason as for the original weighted degree, we also define 
the \textbf{original volume} of a hypargraph and a cluster:
\[\origVol(H) = \sum_{v \in V} \origDeg_\omega(v) \ \ \ \ \ \ \
\origVol(V_i) = \sum_{v \in V_i} \origDeg_\omega(v)\]
This way, the original volume $\origVol$ of a cluster or the whole hypergraph 
in the coarsened hypergraph is equal to the corresponding volume in the initial 
- \textit{original} - hypergraph.


STOPPED here

  \textbf{Conductance of a cut} $S \subseteq V$: 
  \[\mathbf{\varphi(S)} = \frac{\sum_{e \in \partial S}\omega(e)}{\min\{vol(S), vol(V \backslash S)\}}\]

  $k$-way partition of $H$: $V = \sqcup_{i = 1}^{k} V_k$

  \textbf{Conductance of a $k$-way partition}: 
  \[\varphi(V_1, \dots, V_k) = \max_{\emptyset \neq I \subset \{1, \dots, k\}}\{\varphi(\sqcup_{i \in I} V_i)\}\]






\section{Related Work}
\label{sec:related_work}

\section{Implementation}
\label{sec:implementation}

\section{Experimental Results}
\label{sec:experiments}

\section{Conclusion}
\label{sec:conclusion}



\bibliographystyle{plainnat}
\bibliography{references.bib}

\end{document}
